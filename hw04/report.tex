\documentclass[11pt,a4paper]{article}
\usepackage[dvipsnames]{xcolor}
\usepackage{amsmath,tabularx,geometry,graphicx,multirow,tikz,listings,xfrac}

\usetikzlibrary{positioning}
\newdimen\nodeDist
\nodeDist=35mm
\geometry{a4paper, left=20mm, top=20mm}
\graphicspath{{../imgs/}}
\newcommand\sbullet[1][.5]{\mathbin{\vcenter{\hbox{\scalebox{#1}{$\bullet$}}}}}
\newcommand{\circo}{~\raisebox{1pt}{\tikz \draw[line width=0.5pt] circle(1.1pt);}~}

\title{Aprendizagem 2021/22 Homework IV - Group 66}
\author{João Cardoso, 99251. José João Ferreira, 99259}

\begin{document}

\color{darkgray}
\hspace{-8.25mm}
\renewcommand\tabularxcolumn[1]{m{#1}}
\begin{tabularx}{1.09\textwidth} {>{\raggedright\arraybackslash}X >{\centering\arraybackslash}X >{\raggedleft\arraybackslash}X}
  \includegraphics[scale=0.2]{tecnico.pdf} &
  \textbf{Aprendizagem 2022/23} \par \textbf{Homework IV - Group 66} &
  João Cardoso, 99251 \par José João Ferreira, 99259
\end{tabularx}
\renewcommand\tabularxcolumn[1]{p{#1}}
\color{black}

\begin{center}
\textbf{I. Pen-and-paper}
\end{center}

% PROBLEM 1
\begin{flushleft}
\textbf{1)}
\small
\end{flushleft}
\normalsize

% PROBLEM 2
\begin{flushleft}
\textbf{2)}
\par\textbf{a.}
\small
\normalsize
\par\textbf{b.}
\small
\end{flushleft}
\normalsize

\pagebreak
\color{darkgray}
\hspace{-8.25mm}
\renewcommand\tabularxcolumn[1]{m{#1}}
\begin{tabularx}{1.09\textwidth} {>{\raggedright\arraybackslash}X >{\centering\arraybackslash}X >{\raggedleft\arraybackslash}X}
  \includegraphics[scale=0.2]{tecnico.pdf} &
  \textbf{Aprendizagem 2022/23} \par \textbf{Homework IV - Group 66} &
  João Cardoso, 99251 \par José João Ferreira, 99259
\end{tabularx}
\renewcommand\tabularxcolumn[1]{p{#1}}
\color{black}

\begin{center}
\textbf{II. Programming and critical analysis}
\end{center}

% PROBLEM 1
\begin{flushleft}
\textbf{1)}
\end{flushleft}

% PROBLEM 2
\begin{flushleft}
\textbf{2)}
\end{flushleft}

% PROBLEM 3
\begin{flushleft}
\textbf{3)}
\end{flushleft}

% PROBLEM 4
\begin{flushleft}
\textbf{4)}
\end{flushleft}

\pagebreak
\hspace{-8.25mm}
\color{darkgray}
\renewcommand\tabularxcolumn[1]{m{#1}}
\begin{tabularx}{1.09\textwidth} {>{\raggedright\arraybackslash}X >{\centering\arraybackslash}X >{\raggedleft\arraybackslash}X}
  \includegraphics[scale=0.2]{tecnico.pdf} &
  \textbf{Aprendizagem 2022/23} \par \textbf{Homework IV - Group 66} &
  João Cardoso, 99251 \par José João Ferreira, 99259
\end{tabularx}
\renewcommand\tabularxcolumn[1]{p{#1}}
\color{black}

\begin{center}
\textbf{III. Appendix}
\end{center}

\definecolor{backcolour}{rgb}{0.95,0.95,0.92}
\definecolor{codegreen}{rgb}{0,0.6,0}
\definecolor{codepurple}{rgb}{0.58,0,0.82}
\definecolor{codegray}{rgb}{0.5,0.5,0.5}
\definecolor{codered}{rgb}{0.75,0.25,0}
\definecolor{codeyellow}{rgb}{0.55,0.55,0.0}
\definecolor{codeForestGreen}{rgb}{0.05,0.65,0.55}

\lstdefinestyle{mystyle}{
    backgroundcolor=\color{backcolour},
    commentstyle=\color{codegreen},
    keywordstyle=\color{codepurple},
    numberstyle=\tiny\color{codegray},
    stringstyle=\color{codered},
    emph=[0]{loadarff,train_test_split,mean_absolute_error,drop,print,format,abs,boxplot,title,ylabel,show,hist,xlabel},
    emphstyle=[0]\color{codeyellow},
    emph=[1]{pandas,pd,scipy,io,arff,sklearn,model_selection,linear_model,Ridge,neural_network,MLPRegressor,metrics,matplotlib,pyplot,DataFrame,range,dict},
    emphstyle=[1]\color{codeForestGreen},
    basicstyle=\ttfamily\footnotesize,
    breakatwhitespace=false,
    breaklines=true,
    captionpos=b,
    keepspaces=true,
    numbers=left,
    numbersep=7.5pt,
    showspaces=false,
    showstringspaces=false,
    showtabs=false,
    tabsize=2,
    language=Python,
    morekeywords={as}}
\lstset{style=mystyle}

\hspace{-8.25mm}
\begin{tabularx}{1.09\textwidth}{X}
  \lstinputlisting{hw04.py}
\end{tabularx}

\end{document}