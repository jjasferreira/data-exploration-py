\documentclass[11pt,a4paper]{article}
\usepackage[dvipsnames]{xcolor}
\usepackage{amsmath,tabularx,geometry,graphicx,multirow,tikz,listings,xfrac}

\usetikzlibrary{positioning}
\newdimen\nodeDist
\nodeDist=35mm
\geometry{a4paper, left=20mm, top=20mm}
\graphicspath{{../imgs/}}
\renewcommand\tabularxcolumn[1]{m{#1}}

\title{Aprendizagem 2021/22 Homework III - Group 66}
\author{João Cardoso, 99251. José João Ferreira, 99259}

\begin{document}

\color{darkgray}
\hspace{-8.25mm}
\begin{tabularx}{1.09\textwidth} {>{\raggedright\arraybackslash}X >{\centering\arraybackslash}X >{\raggedleft\arraybackslash}X}
  \includegraphics[scale=0.2]{tecnico.pdf} &
  \textbf{Aprendizagem 2022/23} \par \textbf{Homework III - Group 66} &
  João Cardoso, 99251 \par José João Ferreira, 99259
\end{tabularx}
\color{black}

\begin{center}
\textbf{I. Pen-and-paper}
\end{center}

% PROBLEM 1
\begin{flushleft}
\textbf{1)}
\small
\end{flushleft}
\normalsize

% PROBLEM 2
\begin{flushleft}
\textbf{2)}
\small
\end{flushleft}
\normalsize

% PROBLEM 3
\begin{flushleft}
\textbf{3)}
\small
$ a^{[1]} = w^{[1]}x + b^{[1]} $ \hspace{5mm} $ a^{[1]} = w^{[1]}x + b^{[1]} $
\begin{flalign*}
  \fbox{X = 0.8} \\
  Z = 24 \\
  & a^{[1]} = w^{[1]}x + b^{[1]} &&\\
  & h^{[1]} = f(a^{[1]}) &&\\
  & a^{[2]} = w^{[2]}h^{[1]} + b^{[2]} &&\\
  & h^{[2]} = f(a^{[2]}) = \hat{z} &&\\
  h^{[1]} &= \begin{bmatrix} 0.5 \\ 0.5 \end{bmatrix} &&\\
  &\mu = \frac{\sum(y_{3\_i})}{size(v)} = 0.84 &&\\
  &\sigma^2 = \frac{\sum(y_{3\_i} - \mu)^2}{size(v) - 1} = 0.063 &&\\
  &P(y_3 = y_{3\_new} | class = P) = \underline{N(y_{3\_new} | \mu = 0.84, \sigma^2 = 0.063)} &&\\
  \end{flalign*}
\end{flushleft}
\normalsize

% PROBLEM 4
\begin{flushleft}\vspace{2mm}
\textbf{4)}
A
\end{flushleft}

\pagebreak

\color{darkgray}
\hspace{-8.25mm}
\begin{tabularx}{1.09\textwidth} {>{\raggedright\arraybackslash}X >{\centering\arraybackslash}X >{\raggedleft\arraybackslash}X}
  \includegraphics[scale=0.2]{tecnico.pdf} &
  \textbf{Aprendizagem 2022/23} \par \textbf{Homework III - Group 66} &
  João Cardoso, 99251 \par José João Ferreira, 99259
\end{tabularx}
\color{black}

\begin{center}
\textbf{II. Programming and critical analysis}
\end{center}

% PROBLEM 5
\begin{flushleft}
\textbf{5)} \par
\begin{center}
\end{center}
\end{flushleft}
\vspace*{2mm}

\pagebreak

\color{darkgray}
\hspace{-8.25mm}
\begin{tabularx}{1.09\textwidth} {>{\raggedright\arraybackslash}X >{\centering\arraybackslash}X >{\raggedleft\arraybackslash}X}
  \includegraphics[scale=0.2]{tecnico.pdf} &
  \textbf{Aprendizagem 2022/23} \par \textbf{Homework III - Group 66} &
  João Cardoso, 99251 \par José João Ferreira, 99259
\end{tabularx}
\color{black}

\begin{center}
\textbf{III. Appendix}
\end{center}

\definecolor{backcolour}{rgb}{0.95,0.95,0.92}
\definecolor{codegreen}{rgb}{0,0.6,0}
\definecolor{codepurple}{rgb}{0.58,0,0.82}
\definecolor{codegray}{rgb}{0.5,0.5,0.5}
\definecolor{codered}{rgb}{0.75,0.25,0}
\definecolor{codeyellow}{rgb}{0.55,0.55,0.0}
\definecolor{codeForestGreen}{rgb}{0.05,0.65,0.55}

\lstdefinestyle{mystyle}{
    backgroundcolor=\color{backcolour},
    commentstyle=\color{codegreen},
    keywordstyle=\color{codepurple},
    numberstyle=\tiny\color{codegray},
    stringstyle=\color{codered},
    emph=[0]{loadarff,confusion_matrix,accuracy_score,ttest_rel,drop,fit_transform,zeros,split,fit,predict,append,plot,set_title,print},
    emphstyle=[0]\color{codeyellow},
    emph=[1]{scipy,io,arff,pandas,pd,DataFrame,sklearn,preprocessing,numpy,np,LabelEncoder,model_selection,StratifiedKFold,neighbors,KNeighborsClassifier,naive_bayes,GaussianNB,metrics,ConfusionMatrixDisplay,stats},
    emphstyle=[1]\color{codeForestGreen},
    basicstyle=\ttfamily\footnotesize,
    breakatwhitespace=false,
    breaklines=true,
    captionpos=b,
    keepspaces=true,
    numbers=left,
    numbersep=7.5pt,
    showspaces=false,
    showstringspaces=false,
    showtabs=false,
    tabsize=2,
    language=Python,
    morekeywords={as}}
\lstset{style=mystyle}

\end{document}